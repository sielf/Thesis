%% ++++++++++++++++++++++++++++++++++++++++++++++++++++++++++++
%% Kapitel 1: Einleitung, Problemstellung
%% ++++++++++++++++++++++++++++++++++++++++++++++++++++++++++++
%
%  Gerüst:
%  * Version 0.11
%  * Dipl.-Ing. Karsten Renhak, karsten.renhak@tu-ilmenau.de
%  * Fachgebiet Kommunikationsnetze, TU Ilmenau
%
%  Für Hauptseminare, Studienarbeiten, Diplomarbeiten
%
%  Autor           : Max Mustermann
%  Letzte Änderung : 31.12.2011
%

% Hier ein paar Abkürzungen, die ins Abkürzungsverzeichnis
% übernommen werden sollen. Die Buchstaben der Abkürzung können
% in eine \markup{}-Schachtelung geklammert werden, damit sie
% im Verzeichnis besser lesbar sind.
\nomenclature{WWW}{\markup{W}orld \markup{W}ide \markup{W}eb}

\chapter{\LaTeX}
\section{Das Schreiben einer Ausarbeitung mit \LaTeX}
Bei \LaTeX\ schreibt man seinen Text einfach als reinen Text in einem
Texteditor seiner Wahl herunter. Umlaute können direkt als "`äÄöÖüÜß"'
eingegeben werden. Bei Anführungszeichen wird im deutschen zwischen
zwei "`Versionen"' unterschieden. ``Amerikanische'' Anführungszeichen
können natürlich ebenfalls verwendet werden.

Absätze mit neuem Einzug werden durch Freilassung einer Zeile im
Quelltext erzeugt. Dabei ist es egal, ab man eine oder mehrere Leerzeilen
einfügt. Ebenso  ist      es egal ob man             
    im Text Leerzeichen einstreut, die Zeile bis zum Rand vollschreibt
oder nicht. Einen Zeilenumbruch ohne Beginn eines neuen Absatzes
\\
kann man ebenfalls erzwingen, auch wenn dies im Fliesstext nicht immer Sinn ergibt.

Diverse Textauszeichnungen sind möglich, sollten aber konsistent verwendet werden.
So bietet es sich beispielsweise an, ein einheitliches Schema für die Einführung von
\emph{Abkürzungen} (Abk.), wie beispielsweilse \emph{Personal Computer} (PC),
zu verwenden. \textbf{Fette Buchstaben} sind bei Bedarf vorhanden,
{\ttfamily Schreibmaschinenschrift} eignet sich für die Nennung von
Programmnamen. Für URLs bietet sich ein spezielles Kommando an,
wie z.B. \url{http://www.tu-ilmenau.de/kn}.

\nomenclature{URL}{\markup{U}niform \markup{R}esource \markup{L}ocator}

Literaturverweise setzen eine oder mehrere Literaturdatenbanken voraus.
Diese werden als Textdateien mit der Endung {\ttfamily .bib} angelegt und
von \LaTeX\ verarbeitet. Dies kann man beispielsweise in dem gut
geeigneten Nachschlagewerk~\cite{book:latex} nachlesen. Unter
\cite{link:latexkochbuch} findet man ein "`Kochbuch"' für \LaTeX.

Fussnoten sind eine feine Sache, können aber bei zu häufigem Gebrauch
nerven\footnote{Praktisch, stört aber den Lesefluss.}.



\section{Beispiele zur Gliederung: section}
Kein Text\ldots



\subsection{Unterkapitel subsection}
Kein Text\ldots



\subsubsection{Unterkapitel subsubsection}
Kein Text\ldots

\paragraph{Paragraph}
Kein Text\ldots

\subparagraph{Subparagraph}
Kein Text\ldots



\subsection*{Ein Unterkapitel ohne laufende Nummer}
Es macht nicht immer Sinn ein Kapitel oder Unterkapitel mit einer laufenden
Nummer auszustatten. Manchmal soll nur eine Gliederungshilfe
eingefügt werden, ohne aber im Inhaltsverzeichnis aufzutauchen.
Man erreicht dies, indem man ein Sternchen an den Gliederungsbefehl
anhängt.



\subsection{Ein weiteres Unterkapitel}
Kein Text\ldots



\section{Formeln}
\label{text:formeln}
Formeln sind eine Stärke von \LaTeX . Sie können einerseits im Fließtext hinterlegt
werden, was bei kleinen Formeln
wie $ E=mc^2 $ oder bei \(a^2+b^2=c^2\) noch % Beide Notationen sind gleichwertig
gut funktioniert. Bei größeren Formeln und Herleitungen macht es
dagegen Sinn, diese abgesetzt vom Text aufzuführen.

\begin{equation}
U=R\cdot I
\label{formel:ohm}
\end{equation}

\begin{equation}
R=\frac{U}{I}
\end{equation}

Die laufende Nummerierung kann dabei auch unterdrückt werden:

\begin{displaymath}
A\approx\int\limits_{1}^{\infty}\frac{1}{x}\,\mathrm{d}x
\end{displaymath}

Für mehrzeilige Herleitungen eignet sich auch:

\begin{eqnarray}
(x+y)(x-y) & = & x^2-xy+xy-y^2 \\
& = & x^2 - y^2 \\
(x+y)^2 & = & x^2+2xy+y^2
\end{eqnarray}



\section{Listen und Aufzählungen}
Listen und Aufzählungen braucht man öfters, beispielsweise
die so genannten "`Bullet"'-Listen:

\begin{itemize}
\item Erster Punkt
\item Zweiter Punkt
\item Dritter Punkt
\item \begin{itemize}
      \item Erster Unterpunkt mit Startbullet
      \item Zweiter Unterpunkt mit Startbullet
      \end{itemize}
      \begin{itemize}
      \item Erster Unterpunkt ohne Startbullet
      \item Zweiter Unterpunkt ohne Startbullet
      \end{itemize}

\end{itemize}

Echte Aufzählungen sehen so aus.

\begin{enumerate}
\item Erster Punkt
\item Zweiter Punkt
\item Dritter Punkt
\item \begin{enumerate}
      \item Erster Unterpunkt mit übergeordneter Nummer
      \item Zweiter Unterpunkt mit übergeordneter Nummer
      \end{enumerate}
\begin{enumerate}
      \item Erster Unterpunkt ohne übergeordneter Nummer
      \item Zweiter Unterpunkt ohne übergeordneter Nummer
      \end{enumerate}
\end{enumerate}

Aufzählungen eignen sich auch gut zur Gliederung innerhalb
eines Kapitels:

\begin{itemize}
\item\textbf{Argument A:}\\
      Blah\ldots

      \emph{Fazit:}\\
      Funktioniert, weil \ldots
\item\textbf{Argument B:}\\
      Fasel\ldots

      \emph{Fazit:}\\
      Funktioniert nicht, weil \ldots
\end{itemize}

Zudem gibt es auch noch die Description-Umgebung:
\begin{description}
\item[Schlagwort]
      So kann man einzelne Begriffe der Reihe nach einführen und
      dabei auch gleich erklären. Nach einem Zeilenunbruch
      wird eingerückt.
\item[Noch ein Begriff]
      Dabei findet aber keine horizontale Ausrichtung statt.
\end{description}


\section{Querverweise}
\label{text:querverweise}
Ein Dokument kann Querverweise enthalten. Diese können sich unter anderem
auf Grafiken, Tabellen, Formeln oder Absätze beziehen. Der Verweis kann dabei
entweder die Nummerierung des Objektes oder dessen Seitenzahl zurückliefern.
Der aktuelle Abschnitt lautet
beispielsweise \ref{text:querverweise} und beginnt auf Seite \pageref{text:querverweise}.
Die dazu notwendigen "`Anker"' ({\ttfamily labels}) enthalten einen Kenner, welcher zwar
frei wählbar ist, aber aus Gründen der Übersicht nach einem einheitlichen Schema konsistent gebildet werden sollte.

Eine Grafik befindet sich beispielsweise in Kapitel \ref{text:kile}, ihre Bezeichnung
lautet \ref{bild:kile} und zu finden ist Sie auf Seite \pageref{bild:kile}.
Das Ohmsche Gesetz wird in Formel \ref{formel:ohm} auf Seite \pageref{formel:ohm}
wiedergegeben.
