%% ++++++++++++++++++++++++++++++++++++++++++++++++++++++++++++
%% Kapitel 3: Allgemeine Hinweise
%% ++++++++++++++++++++++++++++++++++++++++++++++++++++++++++++
%
%  Gerüst:
%  * Version 0.11
%  * Dipl.-Ing. Karsten Renhak, karsten.renhak@tu-ilmenau.de
%  * Fachgebiet Kommunikationsnetze, TU Ilmenau
%
%  Für Hauptseminare, Studienarbeiten, Diplomarbeiten
%
%  Autor           : Max Mustermann
%  Letzte Änderung : 31.12.2011
%

\chapter{Allgemeine Hinweise}
\section{\LaTeX-bezogen}
\begin{description}
\item[Abkürzungsverzeichnis]
      Sollte das Abkürzungsverzeichnis nach dem Hinzufügen eines
      {\ttfamily nomenclature}"=Kommandos nicht aktualisiert werden,
      muss der {\ttfamily makeindex}"=Aufruf
      manuell in der Konsole gestartet werden. Manche
      Entwicklungsumgebungen machen dies aber schon automatisch.
      Bitte die genannten Parameter nicht vergessen!

      Bei Benutzern der GUI \texttt{Kile} kann es vorkommen,
      dass der \texttt{makeindex}-Befehl nicht automatisch ausgeführt
      wird, scheint ein Bug zu sein. In diesem Fall kann der Index
      auch manuell durch Aufruf von \texttt{makeindex} aktualisiert
      werden.
\item[Thesenpapier]
      Für die Thesen wurde mit der Version 0.8 an ein eigenständiges
      Dokument namens {\ttfamily thesen-handout.tex} hinzugefügt.
      Es bindet ebenso wie das Hauptdokument die Datei
      {\ttfamily thesen.tex} ein, erzeugt aber eben nur dieses
      eine Blatt ohne eine Seitenzahl.
\item[Beidseitiger Druck]
      Im Zentraldokument {\ttfamily dokument.tex} kann das Layout
      auf doppelseitigen Druck umgeschaltet werden (Option
      {\ttfamily twoside} statt {\ttfamily oneside}). Allerdings
      verlangen manche Prüfungsämter explizit einen einseitigen
      Druck! Neue Kapitel ({\ttfamily chapter}) beginnen dabei
      automatisch auf einer Vorderseite (\(\to \) rechte Seite).
      Die Ränder sind dabei innen nur halb so breit wie außen, was
      aber Absicht ist: Zusammen ergeben die linke und die rechte
      Seite innen einen "`weißen Streifen"', der genauso breit ist wie die
      äußeren Ränder.
\item[Überlange Kapitelüberschriften]
      Manchmal müssen Überschriften sehr lang sein, sodass sie von \LaTeX\ 
      umgebrochen werden. Dieses Verhalten ist aber weder im Inhaltsverzeichnis
      noch in der Kopfzeile erwünscht! Daher kann man zu einer überlangen
      Überschrift auch eine Kurzform mit angeben, welche dann im Inhaltsverzeichnis
      und im Dokumentenkopf verwendet wird:\\
      {\ttfamily \textbackslash chapter[Kurzform]\{Langform\}}
\item[Einzüge]
      Bitte \emph{nicht!} die Einzüge ändern oder abschalten. Das ist
      so gewollt und verbessert den Lesefluss! (Stichwort
      \texttt{\textbackslash setlength\textbackslash parindent\{0pt\}})!
\item[BibTeX-Einträge mit mehreren Autoren]
      Sollen mehrere Autoren angegeben erden, so sind diese einzeln
      als \emph{Vorname Nachname} anzugeben und durch \texttt{and}
      voneinander zu trennen. BibTeX ersetzt das \texttt{and} dann
      durch das deutsche "`und"':\\
      \texttt{author = \{Adam Riese and Eva Zwerg\},}
\end{description}




\section{Inhaltlich}
\begin{itemize}
\item Überschriften im Inhaltsverzeichnis nie tiefer als
      vier Ebenen. Dies geht mit \LaTeX\ auch gar nicht anders,
      da {\ttfamily subsubsection} bereits die niedrigste
      Schachtelungstiefe darstellt, welche noch im
      Inhaltsverzeichnis aufgeführt wird.
\item Die Kapitel sollten in der späteren Ausarbeitung anders
      benannt werden als in dieser Formatvorlage. Eine Diplomarbeit
      \emph{kann} beispielsweise aus der folgenden Aufteilung bestehen:

      \begin{enumerate}
      \item Problemstellung
      \item Theoretische Grundlagen
      \item Herleitung
      \item Der Prototyp
      \item Zusammenfassung
      \item Ausblick
      \end{enumerate}
\item Es empfiehlt sich, ein Programm zur Rechtschreibprüfung zu
      installieren. Alternativ zu einer \LaTeX"=fähigen
      Rechtschreibkorrektursoftware kann ein Abschnitt auch
      in bspw. Microsoft Word getippt und geprüft werden, bevor
      er dann in das \LaTeX"=Dokument eingefügt wird.
\item Für Diplomarbeiten wird generell ein englischer "`Abstract"'
      benötigt!
\end{itemize}
