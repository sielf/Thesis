
\chapter{Заключение}
В данной работе была рассмотрена система обобщенной частотной разделения каналов. В первой части мы рассматривали введение в данную систему, метод разделения каналов и тензорную обработку данных. В первой секции была описана необходимость внедрения технологии ОбЧРК для уплотнения поднесущих в системе и уменьшения вне-полосных излучений. Более того подобный подход позволяет регулировать по коэффициенту перекрытия подобную систему и добиться наилучшего результата. Во второй секции была введена тензорная алгебра и ее операции необходимы для использования в дальнейших вычислениях. Была рассмотрена модель тензора $PARATCUK2$ которая была использована для моделирования системы ОбЧРК. При этом тензорная модель $PARATCUK2$ позволила упросить запись модулирующей матрицы для данных записав ее через произведение Хатри-Рао между двумя матрицами, где первая матрица была матрицей заполненной коэффициентами Фурье и вторая матрица была матрицей с импульсными характеристиками для символов на разных временных слотах. Эта модель чрезвычайно схожа с моделью вейвлетов своим разделением на временное и частотное пространство. Таким образом система работает очень близко к пределу определенности между пространствами времени и частоты.  В этом же разделе были рассмотрены особенности влияния коэффициента перекрытия системы ОбЧРК и эффекты перекрытия по частоте с увеличением $\alpha$. Результаты данного влияния на производительность показаны в разделе симуляций. Так же была рассмотрена разница между фильтрами с приподнятым косинусом и корнем из приподнятого косинуса.
Во втором разделе работы была рассмотрена система ОбЧРК в модели с одной передающей и одной принимающей антенной. Мы моделировали переданный сигнал при помощи тензорной модели $PARATUCK2$ в векторизированной форме. Мы рассмотрели два алгоритма применимых в системе ОбЧРК: алгоритм поиска поднесущих частот и алгоритм полу-слепого приемника с оценкой состояния канала. В первом алгоритме мы подразумевали систему с умным передатчиком отключающим работу поднесущих в случае если в ее диапазоне частот присутствует другая излучающая система. При этом приемное устройство должно определить на каких конкретно поднесущих были переданы данные. Для решения данной задачи полагалось знание передатчиком первого символа в блоке данных каждой поднесущей. Таким образом задача имеет одно решение так как количество известных равно количеству неизвестных. Однако поскольку функция нелинейна и зависит от своих же переменных, необходимо решить данную задачу методами оптимизации, что и было сделано в ходе работы.
В дальнейшем был рассмотрен случай оценки канала без циклического префикса используя только набор известных в блоке данных символов. Для этого была поставлена оптимизационная задача выписанная в систему нелинейных уравнений и решенная методом Ньютона. При этом данных подход показал лучшую производительность, чем в случае если использовать методы основанные на обратно фильтрации. К сожалению недостатком данного метода является его вычислительная сложность возрастающая с увеличением размеров блока. Однако для таких методов применимы итеративные методы решения задач позволяющие вычислять вычислительно емкие алгоритмы. К классу подобных алгоритмов принадлежат к примеру алгоритмы "ISTA" и "FISTA".
В последней часты мы расширили описанные в предыдущей секции алгоритмы на случай со множеством передающих и множеством принимающих антенн. Большое количество антенн делает задачу значительно более сложной в силу увеличения количества переменных. Так же усложняется моделирование системы так как с физической точки зрения значительно увеличивается сложность ее описания. Мы переписали выражение описывающее принятые данные векторном виде и таким образом смогли записать канальную модель как произведение блочной треугольной матрицы с вектором символов. Это позволила в дальнейшем упростить выражение и записать подобное равенство для канальных коэффициентов. Тогда при помощи известных для приемника символов была решена задача оценки канала и поиска неизвестных символов для одного блока данных. Алгоритмы работы были сделаны на основе ПМНК и метода Ньютона. Однако оба алгоритма показали недостаточно высокую производительность ,что было связано с высоким собственным числом матриц канала. Так же в будущей работе будет предложено решение данных проблем при помощи методов регуляризации. Кроме того производительность алгоритмов при оценке каналов оказалась достаточно высока, что  говорит о достаточно хороших перспективах работы алгоритма. Так же был расширен до случая МВМВ алгоритм поиска поднесущих, причем он был расширен до случаев когда любая антенная передает свой уникальных набор поднесуших. При этом была решена задача поиска коэффициентов 4 разными  способами: при помощи МНК, алгоритмы основанные на модели ОВОВ и МВМВ а так же алгоритм основанный на разложении тензора на матрицы. При этом использовалось предположение о работе системы в канале с Рэлеевским замиранием канала. Как показали результаты работы системы, алгоритм достаточно сильно ухудшает производительность системы, что связано с большим количеством ошибок в выбранной модели канала. 
В заключение можно сказать, что оптимизационные алгоритмы позволяют увеличить производительность алгоритмов по сравнению с традиционными аналитическими решениями. В будущей работы будет рассмотрена модель МВМВ передачи данных а так же будет выполнена регуляризация оптимизационных методов связанных с плохой обусловленностью задачи.

%
%In the last part we extended the model for the MIMO case. The high number of the antennas make the model very complex for modeling. We extended the vectorized model of the PARATUCK2 with change in one matrix and use the same model for the MIMO case, which simplified new derivations. We developed the subcarrier selection approach for the MIMO. We assumed the same situation as in the SISO case, but used two predefined options: the overall transmisison block is sent for subcarrier analysis, and the semi-blind approach. We developed both algorithms and tested results via simulations. The algorithm allow to estimate the turned on subcarriers automatically without information from the transmitter. The performance of the algorithms is lower than in known case. The performance decrease is explained with the huge number of estimated coefficients. The next algorithm which we assumed is frequency selective channel estimation for the MIMO system. We assumed Rayleigh fading channel with frequency selection for each transmitter-receiver pair. We developed two algorithms based on the Newton optimization and ALS based optimization algorithm.We have made simulations for the all of this algorithms. The simulations doesn't allow to make prediction about performance of the algorithm, due to the complex model of the receiver part. There is one disadvantage of the optimization based algorithms, all of them are computationally expensive. The developed algorithms have this disadvantage and the computational complexity become the very strong problem because the MIMO model increase amount of data significantly. The deeper analysis necessary for the optimization algorithms overview. 
% As the conclusion over all thesis we can say that optimization algorithm allow to increase performance of the system and estimate the symbols in more precise way.
% The more precise estimation of the MIMO frequency selective model will be done in the future work. In the future work we will consider the different channel types in the MIMO model and ways to explain the physical channel in the matrix way. The multiplication between the Rayleigh fading and frequency selective channel is quite complex and must be simplified.
 
% In the second part we considered the GFDM system with one transmit and one receive antenna. We modeled the transmitted signal with the PARATUCK2 model in the vectorized form. We considered the two approached in the GFDM receiver side: subcarrier coefficients estimation approach and channel estimation approach. In the first approach we assumed that transmitter turn off number of subcarriers and doesn't transmit data from them. The reciever know the first symbol on each subcarrier and try to estimate which subcarrier is turned on. We developed algorithm which estimates the symbols and make decision about turned on coefficients. Algorithm allow to use the spectrum sensing in the automatic way at the receiver without predefined subcarriers. The second approach we consider in the model with channel. We assumed that there is frequency selective channel between receiver and transmitter. We developed algorithm, which allow to estimate the channel and the symbols in one algorithm. This algorithm called semi-blind receiver. We have made simulations for the all of this algorithms. The simulations show, that results have good performance. There is one disadvantage of the optimization based algorithms, all of them are computationally expensive. The developed algorithms have the same disadvantage. The disadvantage become significant, if the transmission block size become large. The matrix inversion calculation become the hard task. There is additional iterative approaches so called ISTA" and "FISTA" which allow to decrease computational price. The semi-blind receiver outperforms the FFT receiver with known channel. This result show the high performance of the semi-blind receiver.