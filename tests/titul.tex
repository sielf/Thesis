%\documentclass[a4paper]{article}
%\usepackage[T1,OT2,T2B]{fontenc}
%\usepackage[cp1251]{inputenc}
%\usepackage[14pt]{extsizes} 
%\usepackage[russian]{babel}
%\usepackage{setspace,amsmath}
%\usepackage[left=20mm, top=15mm, right=15mm, bottom=15mm, nohead, footskip=10mm]{geometry} % настройки полей документа
%\usepackage{mathptmx}
%\usepackage{times}
%\begin{document} % начало документа
 
% НАЧАЛО ТИТУЛЬНОГО ЛИСТА
\begin{center}
\hfill \break
\large{МИНОБРНАУКИ РОССИИ}\\
\footnotesize{ФЕДЕРАЛЬНОЕ ГОСУДАРСТВЕННОЕ БЮДЖЕТНОЕ ОБРАЗОВАТЕЛЬНОЕ УЧРЕЖДЕНИЕ}\\ 
\footnotesize{ВЫСШЕГО ПРОФЕССИОНАЛЬНОГО ОБРАЗОВАНИЯ}\\
\small{\textbf{«КАЗАНСКИЙ НАЦИОНАЛЬНЫЙ ИССЛЕДОВАТЕЛЬСКИЙ НАУЧНЫЙ ТЕХНИЧЕСКИЙ УНИВЕРСИТЕТ им. А.Н. Туполева-КАИ»}}\\
\hfill \break
\normalsize{Институт радиоэлектроники и телекоммуникаций}\\
 \hfill \break
\normalsize{Кафедра радиоэлектронных и телекоммуникационных систем}\\
\hfill\break
\normalsize{при сотрудничестве с \\} 
\hfill \break
\textbf{TECHNISCHE UNIVERSITÄT ILMENAU \\
Fakultät für Elektrotechnik und Informationstechnik \\
Fachgebiet "Kommunikationsnetze" \\}
\hfill \break
\hfill \break
\large{МАГИСТЕРСКАЯ ДИССЕРТАЦИЯ\\
Моделирование систем с обобщенным частотным разделением каналов на основе тензорной алгебры}\\
\hfill \break
\normalsize{Направление  110402 Инфокоммуникационные технологии и системы связи
\hfill \break
Магистерская программа  Communication and signal processing  }\\
\hfill \break
\end{center}
\normalsize{ \hspace{28pt} Допущено к защите в ГЭК  16.08.2016} \hfill \break
\hfill \break
 
\normalsize{ 
\begin{tabular}{cccc}
Зав.кафедрой & \underline{\hspace{3cm}} &  д.т.н.,  проф. & Г.И. Щербаков \\\\
Обучающийся & \underline{\hspace{3cm}} & &Б.М.Валеев \\\\
Руководитель & \underline{\hspace{3cm}}& д.т.н., проф.&  Ю.К. Евдокимов \\\\
\end{tabular}
}\\
\hfill \break
\hfill \break
\begin{center} Казань 2016 \end{center}
\thispagestyle{empty} % выключаем отображение номера для этой страницы
 
% КОНЕЦ ТИТУЛЬНОГО ЛИСТА
 
\newpage
 %
%\end{document}  % КОНЕЦ ДОКУМЕНТА !