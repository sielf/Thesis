\chapter{Summary and future work}\label{sec:Conc}
In this work generalized frequency division multiplexing was discussed and investigated.
In the first part is concerned the introduction into the multiplexing scheme and into the tensor based processing. We show the main equations of the tensor based processing in the second section. We discussed the PARATUCK2 model, which become the main model for the transmitted signal. In addition, the PARATUCK2 model allow to simplify the modulation matrix as the Khatri-Rao product between two matrices, where the first is the Fourier transform matrix and the second is time filter matrix for symbols in block. There is one similar model of data explanation, which called wavelets.  In this part was considered the main filters, which are used in the GFDM system. There is strong influence on the system performance from the $roll-off$ factor of the time filter. This influence is shown in the simulation results from second part. We considered the raised cosine and root-raised cosine filters. 

In the second part, we considered the GFDM system with one transmit and one receive antenna. We modelled the transmitted signal with the PARATUCK2 model in the vectorized form. We considered the two approached in the GFDM receiver side: subcarrier coefficients estimation approach and channel estimation approach. In the first approach we assumed that transmitter turn off number of subcarriers and doesn't transmit data from them. The receiver knows the first symbol on each subcarrier and try to estimate which subcarrier is turned on. We developed algorithm which estimates the symbols and make decision about turned on coefficients. Algorithm allow using the spectrum sensing in the automatic way at the receiver without predefined subcarriers. The second approach we consider in the model with channel. We assumed that there is frequency selective channel between receiver and transmitter. We developed algorithm, which allow to estimate the channel and the symbols in one algorithm. This algorithm called semi-blind receiver. We have made simulations for the all of this algorithms. The simulations show, that results have good performance. There is one disadvantage of the optimization based algorithms, all of them are computationally expensive. The developed algorithms have the same disadvantage. The disadvantage become significant, if the transmission block size become large. The matrix inversion calculation become the hard task. There is additional iterative approaches so called ISTA" and "FISTA" which allow decreasing computational price. The semi-blind receiver outperforms the FFT receiver with known channel. This result shows the high performance of the semi-blind receiver.
In the last part we extended the model for the MIMO case. The high number of the antennas make the model very complex for modeling. We developed the subcarrier selection approach for the MIMO. We assumed the same situation as in the SISO case, but used two predefined options: the overall transmission block is sent for subcarrier analysis, and the semi-blind approach. We developed both algorithms and tested results via simulations. The algorithm allow to estimate the turned on subcarriers automatically without information from the transmitter. The performance of the algorithms is lower than in known case. The performance decrease is explained with the huge number of estimated coefficients. The next algorithm which we assumed is frequency selective channel estimation for the MIMO system. We assumed Rayleigh fading channel with frequency selection for each transmitter-receiver pair. We developed two algorithms based on the Newton optimization and ALS based optimization algorithm.We have made simulations for the all of theese algorithms. The simulations don't allow making prediction about performance of the algorithm, due to the complex model of the receiver part. There is one disadvantage of the optimization based algorithms, all of them are computationally expensive. The developed algorithms have this disadvantage and the computational complexity become the very strong problem because the MIMO model increase amount of data significantly. The deeper analysis necessary for the optimization algorithms overview. 
 As the conclusion over all thesis we can say that optimization algorithm allow increasing performance of the system and estimate the symbols in more precise way.
 The more precise estimation of the MIMO frequency selective model will be done in the future work. In the future work we will consider the different channel types in the MIMO model and ways to explain the physical channel. The multiplication between the Rayleigh fading and frequency selective channel is quite complex and must be simplified.